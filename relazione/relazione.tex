\documentclass[openany, a4paper,10pt] {article}
\usepackage[T1]{fontenc}
\usepackage[utf8]{inputenc}
\usepackage[italian]{babel}
\usepackage{indentfirst}

\begin{document}

\begin{center}

\begin{Huge} Programmazione ad Oggetti \end{Huge}

\vfill

\begin{Large} \textbf{Nome:} Follador Saverio \\ \textbf{Matricola:} 1096984 \end{Large}

\end{center}

\newpage

\tableofcontents

\newpage

\section{Introduzione}

\subsection{Ambiente di sviluppo}
\begin{itemize}
	\item \textbf{Sistema Operativo:} Manjaro Linux 16.10
	\item \textbf{Compilatore:} GCC 6.2.1
	\item \textbf{Versione Qt Creator:} 4.2.0
	\item \textbf{Versione Qt:} 5.7.1
\end{itemize}

\subsection{Compilazione}
Per compilare il progetto posizionarsi tramite terminale all'interno della cartella dello stesso. A questo punto, eseguire il comando \textbf{qmake} per generare
il makefile. Eseguire quindi il comando \textbf{make} per avviare la compilazione. A compilazione terminata all'interno della cartella si potrà trovare un file 
eseguibile \textbf{ProgettoP2}.

\subsection{Descrizione}
Come da consegna si è sviluppato un software che permettesse l'accesso da parte di alcune tipologie di utenti ad un database. Il database che si è scelto di 
modellare è un database di prodotti di una qualsiasi azienda. Al database si può accedere come utente o come amministratore. Esistono tre tipologie di utenti (casuale, utilizzatore e rivenditore), ciascuna delle quali può
ricercare prodotti conoscendone il nome e visualizzarne alcune informazioni. Le informazioni visualizzate saranno più o meno complete a seconda della tipologia
di utente. L'amministratore può inserire, eliminare o modificare prodotti e utenti.



\end{document}