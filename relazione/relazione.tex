\documentclass[a4paper,10pt] {article}
\usepackage[T1]{fontenc}
\usepackage[utf8]{inputenc}
\usepackage[italian]{babel}
\usepackage{indentfirst}

\begin{document}

\begin{center}

\begin{Huge} Programmazione ad Oggetti \end{Huge}

\vfill

\begin{Large} \textbf{Nome:} Follador Saverio \\ \textbf{Matricola:} 1096984 \end{Large}

\end{center}

\newpage

\tableofcontents

\newpage

\section{Introduzione}

\subsection{Ambiente di sviluppo}
\begin{itemize}
	\item \textbf{Sistema Operativo:} Manjaro Linux 16.10
	\item \textbf{Compilatore:} GCC 6.3.1
	\item \textbf{Versione Qt Creator:} 4.2.0
	\item \textbf{Versione Qt:} 5.7.1
\end{itemize}

\subsection{Compilazione}
Per compilare il progetto posizionarsi tramite terminale all'interno della cartella dello stesso. A questo punto, eseguire il comando \textbf{qmake} per generare
il makefile. Eseguire quindi il comando \textbf{make} per avviare la compilazione. A compilazione terminata all'interno della cartella si potrà trovare un file 
eseguibile \textbf{ProgettoP2}.

\subsection{Descrizione}
Come da consegna si è sviluppato un software che permettesse l'accesso da parte di alcune tipologie di utenti ad un database. Il database che si è scelto di 
modellare è un database di prodotti di una qualsiasi azienda. Al database si può accedere come utente o come amministratore. Esistono tre tipologie di utenti (casuale, utilizzatore e rivenditore), ciascuna delle quali può
ricercare prodotti conoscendone il nome e visualizzarne alcune informazioni. Le informazioni visualizzate saranno più o meno complete a seconda della tipologia
di utente. L'amministratore può inserire, eliminare o modificare prodotti e utenti.

\newpage

\section{Struttura}

\subsection{Informazioni generali}
Per modellare i database di Utenti e Prodotti si è utilizzata la struttura dati \textbf{vector<T>} contenuta nella libreria STL. Entrambi i database sono formati
da un vector di puntatori a oggetti del tipo rispettivo. Non sono stati utilizzati puntatori smart in quanto, data la tipologia di database modellati, non vengono
mai effettuate copie di oggetti. La gestione del garbage è affidata alle singole funzioni di eliminazione.

\subsection{Classi modellate}
\subsubsection{Utente}
Gli utenti a cui viene garantito l'accesso al database sono gestiti da una gerarchia di classi. Dalla classe base astratta \textbf{Utente} derivano le classi 
\textbf{UtenteCasuale}, \textbf{UtenteUtilizzatore} e \textbf{UtenteRivenditore}. UtenteCasuale rappresenta un utente che si presuppone acceda saltuariamente al
database dell'azienda. Pertanto ha accesso solo ad alcune informazioni dei prodotti con le sue ricerche (nome e uso del prodotto). UtenteUtilizzatore rappresenta
un utente che utilizza i prodotti dell'azienda. In quanto utilizzatore ha accesso anche alla durata dei prodotti oltre che a nome e uso. UtenteRivenditore
rappresenta un rivenditore dei prodotti dell'azienda. Ha accesso a nome, uso, durata e prezzo dei prodotti.\\
La classe base Utente contiene le informazioni base di ciascun utente; le immagazzina attraverso oggetti di due classi create ad hoc: \textbf{LoginPw} e \textbf{Info}.\\
I diversi privilegi di ricerca per i vari tipi di utenti sono implementati tramite funtori (oggetti della classe \textbf{Funtore}).

\subsubsection{Database}
I database sono modellati tramite due classi: \textbf{DatabaseProdotti} e \textbf{DatabaseUtenti}. Salvo alcune piccole variazioni sono due classi speculari, contententi nella
parte privata il contenitore vector di puntatori e nella parte pubblica i metodi per la gestione del database. La classe DatabaseUtenti contiene inoltre un metodo
per l'autenticazione dell'utente.

\subsubsection{Salvataggio su file}

\subsubsection{Controller}
Il controller si occupa di collegare la parte logica e la parte grafica. Sono presenti due classi a questo scopo: \textbf{ControllerAdmin} e \textbf{ControllerUtente}.
Un oggetto controller viene creato ogni qualvolta si effettua l'accesso. Il ControllerAdmin possiede metodi per inserire, modificare ed eliminare oggetti di
entrambi i database; il ControllerUtente possiede metodi solo per interrogare il database dei prodotti.

\subsubsection{Divisione Grafica/Logica}
Si è cercato di seguire l'architettura \textbf{MVC} (Model-View-Controller). La parte grafica è separata dalla parte logica; entrambe vengono messe in comunicazione
fra loro tramite il controller.\\
Ciascuna schermata del progetto è identificata da una specifica classe



\end{document}